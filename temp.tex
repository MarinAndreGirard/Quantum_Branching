\documentclass{article}
\usepackage{amsfonts}
\usepackage{braket}
\usepackage{amsmath}
\usepackage{bbm}
\usepackage{graphicx} % Required for inserting images
\usepackage[margin=2cm]{geometry} % Adjust the value of "2cm" to your desired margin size
\usepackage{subcaption}



\title{Branching dynamics}
\author{marin.girard }
\date{February 2024}

\begin{document}

\maketitle

\section{Introduction}

\textbf{To run your own simulations varying the parameters, or see other sets of paremeters already simulated see \hyperlink{http://github.com/MarinAndreGirard/Quantum_Branching}}

After establishing what we are working with, we make one section for each of the following driving questions

\paragraph{Does Branching dynamics split the energy eigenspace in a particular way?}
Do the worlds share it equally? Proportional to their respective probabilities?

\paragraph{Having established that they share the eigenstates weirdly. Why?}
What drives this dynamics? Does it generalize beyond the ACL model in which we observe this dynamics?

\paragraph{Can we get it all from the energy spectrum and the distribution of the global state in it?}

Energy spectrum + state defined in energy basis -> sharing of Hilbert space by worlds -> explain the phenomena we observed

\section{Adapted ACl model}

We consider system $\mathcal{S}$ + environment $\mathcal{E}$ , where the system is a truncated SHO.
Hamiltonian,
\begin{equation}    
    H=H_s\otimes\mathbbm{1}+\mathbbm{1}\otimes H_e+E_IH_I
\end{equation}
We choose no self interaction for the truncated SHO, so $H_s=\mathbbm{1}$, and we choose a random hermitian matrix for the environment self interaction $H_e$. The interaction Hamiltonian is $H_I=H_q\otimes H_e'$, with $(H_q)_{ij}=q_i\delta_{ij}$ and $H_e'$ another random hermitian matrix. This makes it so that the environment acts on the system depending on its state. \\

\subsection{Schmidt states and pointer states}

We use Schmidt states as our object of study, since they are known to converge to Pointer states, which are an attempt at defining classical Branches of the wavefunction.

Given a system and an environment $\mathcal{S} + \mathcal{E}$ and a pure state defined over both $|\psi(t)\rangle$. It can always be decomposed in its Schmidt basis,
\begin{equation}
    |\psi(t)\rangle = \sum_i \sqrt{p_i(t)}|a_i(t)\rangle|b_i(t)\rangle
\end{equation}
The important points being that the sum is over one index, that the sum is over dimension of the smaller of both systems ($\mathcal{S}$ or $\mathcal{E}$) at most, and that the basis itself depends on the state, which implies basis varying in time.

Decoherence has it that $\langle b_j(t)|b_i(t)\rangle\rightarrow \delta_{ij}$ for $t\rightarrow \infty$. 
Pure decoherence, has the evolution of the system be,
\begin{equation}
    U(t)=\sum_j|a_j\rangle\langle a_j|\otimes U^{\mathcal{E}}_j(t)
\end{equation}
Starting the system in $|b\rangle$, then $U_j^{\mathcal{E}}|b>=|b_j(t)\rangle$. This means that the reduced density matrix of the system $\mathcal{S}$, evovlves as,
\begin{equation}
    \rho_{ij} = \langle b_i(t)|b_i(t)\rangle \rho_{ij}(0)
\end{equation}
This results in the evolution of the system density matrix to a diagonal form. This diagonal form, explains the apparent classicality of the unique measurment result. The 2 results, after some decoherence time, belong to 2 different orthogonal states. 

\subsection{Schmidt states in the ACL}

For $|q_i>$ the eigenstates of $H_q$ and $|e_i>$ the eigenstates of $H_e$, we choose our initial state as being,
\begin{equation}
    |\psi(0)\rangle=(\frac{|q_1\rangle}{\sqrt{w}}+\frac{|q_2\rangle}{\sqrt{1-w}})|e_i\rangle
\end{equation}
which is a separable state.

Then,
\begin{equation}
    H|\psi(0)\rangle=|q_1\rangle(\frac{2+q_1}{\sqrt{w}})\sum_jc_j|e_j\rangle+|q_2\rangle(\frac{2+q_2}{\sqrt{1-w}})\sum_jc_j'|e_j\rangle
\end{equation}
which is no longer separable, and by its factorizability by $|q_1\rangle$ and $|q_2\rangle$, we understand that for all $t$,
\begin{equation}
    |\psi(t)\rangle=\frac{1}{\sqrt{N}}(s_1|q_1\rangle|E_1(t)\rangle+s_2|q_2\rangle|E_2(t)\rangle)
\end{equation}
This means that the Schmidt decomposition of our system+env always has 2 nonzeros schmidt weights at most. Note, $|E_i\rangle=\sum_jc_{ji}(t)|e_j\rangle$.

\subsection{Characterizing the spectrum of the Hamiltonian}
The Hamiltonian who's spectrum we are trying to characterize is composed of 3 parts, $\mathbbm{1}, \mathbbm{1}\otimes H_e,E_IH_q\otimes H_e'$. We are interested in the shape that the distribution of eigenvalues will take. $\mathbbm{1}$ will simply shift that distribution by +1, since it commutes with the other 2. \\

We will use the fact that the random matrices that appear in these Hamiltonians, are Gaussian Unitary Ensemble (GUE) matrices, and are known to have semi-circle spectrum distribution.
Wigner semi-circle law:
\begin{equation}
    f(E)=\frac{2}{\pi R^2}\sqrt{E^2-R^2}
\end{equation}
distribution between $[-R,R]$ bounds.\\

Considering $\mathbbm{1}\otimes H_{e}$, $H_{e}$ has spectrum $\sigma_2=\{\lambda_1,\lambda_2,...\lambda_{d_1}\}$ and the identity is here defined over Hilbert space of dimension $d_1$. This makes  the final spectrum $\sigma_{1,2}={\lambda_1, \lambda_1, ..., \lambda_1, \lambda_2, ...,\lambda_2, \lambda_3, ...}$, where each element of the spectrum is repeated $d_1$ times. The distribution of the spectrum of the first term is then an elevated Wigner semi-circle.
\begin{equation}
    f_{env}(E) = \frac{2}{\pi R^2}\sqrt{R^2-E^2} + 10
\end{equation}

Considering $H_q\otimes H_{e}'$, with the spectrum of the random matrix $\sigma_2=\{\lambda_1',\lambda_2',...\lambda_{d_1}'\}$. We get a new spectrum,  $\sigma_{1,2}=0*\sigma_2\bigcup\sigma_2\bigcup2\sigma_2\bigcup ...\bigcup(d_1-1)\sigma_2$ ie $\sigma_{1,2}= \{0,...,0,\lambda_1', \lambda_2', ..., 2\lambda_1', 2\lambda_2',..., (d_1-1)\lambda_1',...\}$. This produces a sum of Wigner semi-circles with $R_i=i*R_1$ for $i=0,1..,d-1$.
\begin{equation}
    f_{int}(E) = E_I\sum_{j=1}^{d_1-1}\frac{2}{\pi j^2R^2}\sqrt{j^2R^2-E^2} + \delta_{E,0}d_2
\end{equation}
The $\delta_{E,0}d_2$ comes from our SHO having ground state energy $E_0=0$. The $E_I$ factor is added to the distribution here. Since these distributions are centered on 0, the factor $E_I$ acts as a spread factor. (ie the max and min are split $2E_I*R_{effective}$)\\

We verify that we do indeed get this in Fig.~(\ref{fig:wigner_sum}). 

\begin{figure}
    \centering
    \includegraphics[scale=0.5]{Figures/Wigner_sum.png}
    \caption{The set of wigner semi-circles who's sum gives us the spectrum of $H_I$}
    \label{fig:wigner_sum}
\end{figure}

With our simulation we can also get them numerically as seen in Fig.~(\ref{fig:spectra}), we choose $E_I=0.06$ and $w=0.3$.
\begin{figure}
    \centering
    \includegraphics[scale = 0.5]{Figures/Spectra.png}
    \caption{First is the distribution of the state in the total energy eigenbasis. Second is the spectrum of that total energy eigenbasis. Third is the distribution of ythe spectrum of $\mathbbm{1}\otimes H_e$ in its own energy eigenbasis. We do see a Wigner semi-circle lifted up by 10. Finally is the distribution of the spectrum of $H_q\otimes H_e'$, which also matches a sum of 10 Wigner semi-circles with growing $R$. We also see the $200*\delta_{E,0}$ predicted at the center. Note that this last distribution has to be scaled down by the factor $E_I=0.06$ (in this case)}
    \label{fig:spectra}
\end{figure}

About Fig.~(\ref{fig:spectra}), it looks like $\sigma_{total}=\sigma_{env}+E_I\sigma_{int} + 1$ (+1 from identity matrix). But that would only be the case if both Hamiltonians where diagonalizable in a common basis, which they are not, being different independant random matrices. Is it still possible to consider the final spectrum as such? I do not see another way to characterize it.\\

If we take the summing of the spectra as a correct approximation, we have characterized the shape of the total spectrum depending on $E_I$. 
\begin{equation}
    f_{total}(E) = \frac{2}{\pi R^2}\sqrt{R^2-E^2} + E_I\sum_{j=1}^{d_1-1}\frac{2}{\pi j^2R^2}\sqrt{j^2R^2-E^2} + \delta_{E,0}d_2 + 10\theta(R^2-E^2)
\end{equation}
$\theta(x)$ is the Heaviside function. For the $H_{rand, d=200}$ $R\approx 10$.\\

Potential change to the model. The identity spectrum does not add anything interesting we could remove it. It might also be beneficial to make the GS of the SHO different from 0. 

\subsection{Characterizing the spectrum of the state}

Could the high differentiation of where they have probabilities be normal? They are not defined on a lot of dog of the space. Is that important at all? Do they have nonzero def on all the same? Are there dog where s1 is defined that s2 is not defined on?  All of this arcs back to understanding the basics….  I don’t really get why they are not defined on all eigenspaces…

\paragraph{Impact of $E_I$: }Varying $E_I$ changes the total energy eigenbasis. 

\paragraph{Impact of $w$: }The $w$ factor is the probabilities, a factor in front of the 2 branches, so changes the importance of each branches when we vary it. This then impact our metrics, as seen in  Fig.~(\ref{fig:simi}).
\begin{figure}[h]
    \centering
    \includegraphics[scale=0.4]{Figures/Screenshot 2024-02-29 at 14.20.36.png}
    \caption{Evolution of cosine similarity between s1 and s2 over the whole Hilbert space.}
    \label{fig:simi}
\end{figure}
What Fig.~(\ref{fig:simi}) this tells us is that the probability distribution of s1 and s2 evolve to be more different over time. And importantly that $w$ has an impact on how fast that happens, how high the similarity goes and the stability of the low probability in time.




\section{Does Branching dynamics split the energy eigenspace in a particular way?}

This was spurred by the interesting behavior of $Neff_{schmidt,i}/Neff_{total}=\delta_i$, with $Neff_{state}=\frac{1}{\sum_ip_i^2}$, $p_i$ the probability of finding the state we are considering in total energy eigenstate $i$. See Fig.~(\ref{fig:Neff_andy}). The idea of $Neff$ being to measure the spread of the Schmidts over the energy eigenspace. We see this because Neff is maximized by a uniform distribution and minimized by $p_i=\delta_ai$ for some integer $a$. $\delta_i$ is going to depend on how the Schmidt $i$ evolves in the energy eigenspace, which is what we want to study.

\begin{figure}
    \centering
    \includegraphics[scale = 0.4]{Figures/Neff_andy.png}
    \caption{In this specific run, the curves join and stay together. With different parameters, we see the 2 curves stabilize with some distance, positive or negative... the behavior is not yet understood}
    \label{fig:Neff_andy}
\end{figure}

What we saw looking at the graphs for multiple runs of the simulation with varied parameters, is that sometimes both $\delta_i$'s would match up, sometimes after decoherence $\delta_1>\delta_2$ or $\delta_2>\delta_1$.

To try and characterize it we run the simulation for a number of parameters.

It would be interesting to characterize $Neff$ to understand its behavior. Are there necessary conditions for the convergence of the 2 curves? Why does Neff have different behaviors for different $E_I$, $w$ or the initial environment state? 



ADD SUPER GRAPH WITH FULL CHARACTERIZATION OF NEFF  
\begin{figure}
    \centering
    \includegraphics{}
    \caption{}
    \label{fig:characteriazation_Neff}
\end{figure}

We see in Fig.~(\ref{fig:chacteriazation_Neff}) that $\delta_1$ always starts at 1 since it is the global state at $t=0$. It then varies a little from its base value to some equilibrium value slightly above or under 1. This relfects the fact that the distribution shape of Schmidt 1 stays relatively constant. 
$\delta_2$ varies in value (from initial to equ) by a factor from 10 to 100.

Whatever i do, i cannot seem to make $\delta_2$ go above $\delta_1$ in the equilibrium regime. That is, unless I inverse the probabilities. 
Clearly the probability .....
I was wrong, having high w, did not change it....
impact of E_spacing.

\subsection{Other hints that eigenspace is shared in an interesting way}

Looking at the difference between how things are a little after interaction starts and after all has settled, Fig.~(\ref{fig:gif_compare}) we see that the Schmidt states seem to have shared the energy eigenspace.

\begin{figure}[h!]
  \centering
  \begin{subfigure}[b]{0.4\linewidth}
    \includegraphics[width=\linewidth]{Figures/g_1.png}
    \caption{Zoomed view of the probability distribution of Schmidt 1 and 2 in the total energy eigenbasis at frame 11/100}
    \label{fig:1}
  \end{subfigure}
  \begin{subfigure}[b]{0.4\linewidth}
    \includegraphics[width=\linewidth]{Figures/g_2.png}
    \caption{Zoomed view of the probability distribution of Schmidt 1 and 2 in the total energy eigenbasis at frame 89/100}
    \label{fig:2}
  \end{subfigure}
  \caption{There is less large probability overlap at frame 89 than 11}
  \label{fig:gif_compare}
\end{figure}

\subsection{Other metrics to describe the sharing of eigenspace and }
We want to confirm that there is indeed interesting dynamics to find in the sharing of energy eigenspace+we want the right tool to characterize it.

MAKE A FULL GRAPH OF GRAPHS OF EACH METRIC
PUT A FEW COMMENTS ON EACH METRIC


\paragraph{Mean and standard deviation}
To characterize the shape of the disrtributions, we look at the variation of the mean and standard deviation of both distributions in time
MEAN AND SDRD DEV graphs
Note hte difficulty to characterize their behavior easily.... might be hinting at something....
What looking at the gifs seems to tell us is that Schmidt 1 starting with the shape of the global state looses, varies a bit in shape when Schmidt 2 goes from uniformly spread to closely resembling the distribution of Schmidt 1. Looking closer at a few eigenstates, Schmidt2 seems to displace Schmidt 1 from some 
We confirme that by looking at this graph of the means and standard dev.

\paragraph{Overlap of the probability distributions}
overlap

\paragraph{Interference}
Interference

\paragraph{Similarity between s1 and s2}
Similarity between s1 and s2, 2 metrics

\paragraph{Measure of fluctuation in time of the distributions}
Similarity in time, a sort of measure of fluctuation. Maybe useful to characterize the behaviors of s1 and s2 distribution themselves.

\paragraph{Gifs}
Gifs of how the distributions of both Schmidts changes in time.

\paragraph{Occupation of the Hilbert space}
Occupation measure of the hilbert space. 


\subsubsection{General characterization of these metrics}

We generally see these metrics vary in unision to the amount of interference between Schmidt 1 and 2.
we also see an inversiomn in behavior under a change of w to 1-w. 

As double proof of this simple law, is the plots of occupation with w=0.03 and 0.07 (which also have underneath the plots of interference, and we clearly see the same shape.)












Does Schmidt 1 loose in $\delta$. 


Plot of standard deviation. does Schmidt 1 loose in standard deviation? 


Plot of occupation

Plot of overlap.

ie maybe make a nice all_at_once thing.






The distribution of schmidt 1 is "full", ie it starts as the distribution of the global state. As the second Schmidt value goes from 0 to not 0, Schmidt 2 gains in importance, it takes over some of the energy eigenstates on which Schmidt 1 is defined and takes a similar shape (since the shape I think will determine the macro and they both should have essentially the same macro.). So we have 2 distrib of the same shape but not defined over the same space.


\subsection{Are probabilities not just shared in an equal manner?}
Since we have unitary evolution, wont the probability of being in an energy eigenstate for both Schmidt states just sum up to what it is for the global state? They wont because of interferences.

At some time t:
$$|\psi\rangle=\sqrt{s_0}|q_1\rangle|E_1\rangle>+\sqrt{s1}|q_2\rangle|E_2\rangle$$

Taking a total energy eigenstate $|Ai\rangle$

The probability that the state be in this energy eigenstate is 
$$P(|\psi\rangle \text{ in } |A_i\rangle)=|\langle A_i|\psi\rangle|^2 = s_0|\langle A_i|q_1E_1\rangle|^2+s1|\langle A_i|q_2E_2\rangle|^2+\sqrt{s0s1}(\langle A_i|q_1E_1\rangle^*+\langle A_i|q_2E_2\rangle^*)$$

This can also be written,
$$P(|\psi\rangle \text{ in } |A_i\rangle)= s_0P(|Schmidt_1\rangle\text{ in }|A_i\rangle)+ s_1P(|Schmidt_2\rangle\text{ in }|A_i\rangle) +\sqrt{s0s1}(\langle A_i|q_1E_1\rangle^*+\langle A_i|q_2E_2\rangle^*)$$
Where the leftover are quantum interference terms.
Interestingly, $\sqrt{s_0s_1}$ is maximal for $s_0=s_1$ at maximal entanglement, which goes against at least my intuition that the  worlds are supposed to be at maximal non-interference then.

We can visualize this interference in Fig.~(\ref{fig:interf1002}) and Fig.~(\ref{fig:total_interference}).
\begin{figure}
    \centering
    \includegraphics[scale = 0.5]{Figures/Interference_1002.png}
    \caption{In this plot we see the importance of interference effects for the value of probability that a Schmidt has to be in an energy eigenstate. (Chosen for its high interference.)}
    \label{fig:interf1002}
\end{figure}

\begin{figure}
    \centering
    \includegraphics[scale = 0.5]{Figures/total_interf.png}
    \caption{Graph of the total interference between Schmidt states}
    \label{fig:total_interference}
\end{figure}

\section{Having established that they share the eigenstates weirdly. Why?}

Is there some comparison to a case of two boxes of gas with the same pressure and temperature but totally distinct microscopic form?

What explains this way of sharing the eigenspace?
Inrterferences between Schmidt states? 

\paragraph{Explaining the behaviors of our metrics for space sharing.}

Could it just be a case of Schmidt 1 and 2 are fighting over the same space and by interference, they displace each other. And that the interference being proportional to sqrt(s1s2), we have that


\section{Can we get it all from the energy spectrum and the distribution of the global state in it?}

Mad-Dog Everettianism.

Energy spectrum + state defined in energy basis -> sharing of Hilbert space by worlds -> explain the phenomenas we observed

Properties are shared like "this" because the spectrum is like "that"



\section{Other}


We have a certain sharing of the Hilbert space under measurement
This sharing implies certain difference in behavior for quantities such as Neff. And if possible, it would be amazing to recover this sharing from the shape of the spectrum and the distribution of the global state!  mad dog evrietian style.

Something that troubles me. q and w have a large impact on the  “shape “of the distribution. The way I used to see things. Is that the shape of the distribution was the determining factor to how the world would be, ie it would define it classically, ie macroscopically. And to me the 2 worlds couldn’t me so different. But, this macro looking factor (distribution shape), might just be irrelevant.  

Could there be something to say about the fact that we are looking at overlap in the probability space? This is because of course the 2 worlds will be orthogonal in the Hilbert space. 
So our fundamental object of study is really their distribution in the total energy eigenbasis.



Pointer states: found using predicatibility sieve. They are the states that devellope the least entanglement with the environment in a robust manner.

This dynamics of Schmdit basis states towards the pointer states might be worth studying. In the case of no self interaction for the system, the pointer states are eigenstates of the interaction Hamiltonian. Might be interesting to check.

Note the Schmidt basis does not have to be orthogonal.
we have always orthogonal schmidt basis in our ACL case due to orthogonal system state + no self interaction.

(Andreas Albrecht study of Schmidt states dynamics under measurement:
https://journals.aps.org/prd/abstract/10.1103/PhysRevD.46.5504
https://journals.aps.org/prd/abstract/10.1103/PhysRevD.48.3768)

\subsubsection{Distinction between Schmidt and Pointers}
https://journals.aps.org/prd/abstract/10.1103/PhysRevD.40.1071



\subsubsection{A discussion of Schmidt might not be enough}
How pointers might not really be branches https://arxiv.org/pdf/gr-qc/9610028.pdf
Is a paper, looking at the possibility of s=using the schmidt decomposition to seelct the physics set of consistent histories. The find is that it would not be enought.
The schmidt basis is induced by ...

The pointer basis is induced by decoherence, or more precisely is einselected by the environment-system interaction

issue, isn't einselection only defined for S+M+E? while we are working here with S+E?

looking at the schmidt states has the nice property of always looking at 2 orthogonal states.

Are pointer states worlds? https://arxiv.org/abs/gr-qc/9610028 says that no




\end{document}
